\documentclass[10pt,a4paper]{book}
\usepackage{graphicx}
  % Requires \usepackage{graphicx}

\begin{document}
\footnotesize

\begin{flushleft}
  \textsf{\textbf{510 $|$ Introduction to Automata Theory, Formal Languages and Computation}}
\end{flushleft}

\begin{flushleft}
\begin{figure}[h]
  % Requires \usepackage{graphicx}
  \includegraphics[width=3.5cm]{1}\quad Prove that the problem ‘whether L(M) contains two
   
  \qquad\quad\qquad\qquad\qquad\qquad\,strings of the same length’ is undecidable.
\end{figure}
\end{flushleft}

\textbf{\textit{\!\!\!\!\!\!\!\!\!\!Solution:}} This can be proved by reducing the halting problem to it. Reduce the halting problem to this problem by constructing a machine $M_{w}$ using the following process.

On input w to M (the halting problem decider)

\quad

i)\;$M_{w}$ copies w on some special location of its input tape and then performs the same

\quad\,computation as M using the input w written on its input tape.

ii)\;If w = a or w = b, M enters into one of its halting states and $M_{w}$ reaches its final
 
\quad\,state. Thus,L($M_{w}$) = {a, b}

iii)\;Otherwise, reject. Thus L($M_{w}$) = ${\phi}$.

The final decider is given in the following diagram.

\begin{figure}[h]
  \centering
  % Requires \usepackage{graphicx}
  \includegraphics[width=10cm]{2}\\
\end{figure}

\!\!\!Already,it is proved that the halting problem is undecidable.So,$ME_{ql}$ is also undecidable.

\begin{flushleft}
\begin{figure}[h]
  % Requires \usepackage{graphicx}
  \includegraphics[width=3.5cm]{3}\quad Prove that the virus detection is undecidable.
\end{figure}
\end{flushleft}

\textbf{\textit{\!\!\!\!\!\!\!\!\!\!Solution:}} A computer virus is an executable program that can replicate itself and spread from one
computer to another. Every file or program that becomes infected can act as a virus itself, allowing it to spread to other files and computers. But the virus detection problem is undecidable! This can be proved
using the undecidability of halting problem.

Let us define a virus by an algorithm called the ‘is-virus?’

\quad

For a program P,

define (halts for input P)

\qquad\qquad if (is-virus?)\quad\qquad //Execute P assuming P is not a virus.

\qquad\qquad $\{$

\qquad\qquad\; //\, If true


\qquad\qquad\; Virus-code is evaluated. If it is a virus, and P must halt.

\qquad\qquad $\}$

\qquad\qquad\; //\, If false

\qquad\qquad\; Virus-code never executed. Hence, P must not halt.

\qquad

Already, we have proved that the halting problem is undecidable. Hence, the virus detection problem
is undecidable.

\end{document} 